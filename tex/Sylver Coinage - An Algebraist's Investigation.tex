%%%%%%%%%%%%%%%%%%%%%%%%%%%%%%%%%%%%%%%%%
% Beamer Presentation
% LaTeX Template
% Version 1.0 (10/11/12)
%
% This template has been downloaded from:
% http://www.LaTeXTemplates.com
%
% License:
% CC BY-NC-SA 3.0 (http://creativecommons.org/licenses/by-nc-sa/3.0/)
%
%%%%%%%%%%%%%%%%%%%%%%%%%%%%%%%%%%%%%%%%%

%----------------------------------------------------------------------------------------
%	PACKAGES AND THEMES
%----------------------------------------------------------------------------------------

\documentclass{beamer}

\mode<presentation> {

\usetheme{CambridgeUS}

\usecolortheme{dolphin}
}

\usepackage{graphicx} % Allows including images
\usepackage{booktabs} % Allows the use of \toprule, \midrule and \bottomrule in tables

\usepackage[mathscr]{euscript}

%----------------------------------------------------------------------------------------
%	TITLE PAGE
%----------------------------------------------------------------------------------------

\title[Sylver Coinage]{Sylver Coinage\\An Algebraist's Investigation} % The short title appears at the bottom of every slide, the full title is only on the title page

\author{Jeremy L Thompson} % Your name
\institute[USAFA] % Your institution as it will appear on the bottom of every slide, may be shorthand to save space
{United States Air Force Academy \\ % Your institution for the title page
\medskip
\textit{jeremy.thompson@usafa.edu} % Your email address
}
\date{\today} % Date, can be changed to a custom date

\begin{document}

\begin{frame}
\titlepage % Print the title page as the first slide
\end{frame}

%------------------------------------------------

\begin{frame}
\begin{center}
\frametitle{Overview}

Sylver Coinage is a two-player game in which the players alternate choosing natural numbers.\\

~\\

In any turn the player cannot name any number that can be represented as a linear combination of the previously named numbers using non-negative coefficients. The player that must choose the number 1 loses.\\

~\\

In this talk, we investigate what is known about Sylver Coinage as well as how the study of numerical semigroups can inform our strategy.

\end{center}
\end{frame}
 
%------------------------------------------------

\begin{frame}
\frametitle{Overview} % Table of contents slide, comment this block out to remove it
\tableofcontents % Throughout your presentation, if you choose to use \section{} and \subsection{} commands, these will automatically be printed on this slide as an overview of your presentation
\end{frame}

%----------------------------------------------------------------------------------------
%	PRESENTATION SLIDES
%----------------------------------------------------------------------------------------

%------------------------------------------------
\section{Sylver Coinage Game}
%------------------------------------------------

\begin{frame}
\begin{center}
\frametitle{Sylver Coinage Game}

Sylver Coinage is a two-player game in which the players alternate choosing natural numbers\\

~\\

In any turn, the player cannot name any number that can be represented as a linear combination of the previously named numbers using non-negative coefficients\\

~\\

The player that must choose the number 1 loses

\end{center}
\end{frame}

%------------------------------------------------

\begin{frame}
\begin{center}
\frametitle{Example}

Player 1: $10$\\

~\\

M = $\lbrace 10 \rbrace$\\

~\\

~\\

~\\

~\\

~\\

~\\

{\small ${\bf 0}, {\color{gray} 1}, {\color{gray} 2}, {\color{gray} 3}, {\color{gray} 4}, {\color{gray} 5}, {\color{gray} 6}, {\color{gray} 7}, {\color{gray} 8}, {\color{gray} 9}, {\bf 10}, {\color{gray} 11}, {\color{gray} 12}, {\color{gray} 13}, {\color{gray} 14}, {\color{gray} 15}, {\color{gray} 16}, {\color{gray} 17}, {\color{gray} 18}, {\color{gray} 19}, {\bf 20}, {\color{gray} 21}, {\color{gray} 22}, {\color{gray} 23}, {\color{gray} 24}, {\color{gray} 25}, \cdots$\\}

~

\end{center}
\end{frame}

%------------------------------------------------

\begin{frame}
\begin{center}
\frametitle{Example}

Player 2: $5$\\

~\\

M = $\lbrace 10, 5 \rbrace$\\

~\\

~\\

~\\

~\\

~\\

~\\

{\small ${\bf 0}, {\color{gray} 1}, {\color{gray} 2}, {\color{gray} 3}, {\color{gray} 4}, {\bf 5}, {\color{gray} 6}, {\color{gray} 7}, {\color{gray} 8}, {\color{gray} 9}, {\bf 10}, {\color{gray} 11}, {\color{gray} 12}, {\color{gray} 13}, {\color{gray} 14}, {\bf 15}, {\color{gray} 16}, {\color{gray} 17}, {\color{gray} 18}, {\color{gray} 19}, {\bf 20}, {\color{gray} 21}, {\color{gray} 22}, {\color{gray} 23}, {\color{gray} 24}, {\bf 25}, \cdots$\\}

~

\end{center}
\end{frame}

%------------------------------------------------

\begin{frame}
\begin{center}
\frametitle{Example}

Player 1: $7$\\

~\\

M = $\lbrace 10, 5, 7 \rbrace$\\

~\\

~\\

~\\

~\\

~\\

~\\

{\small ${\bf 0}, {\color{gray} 1}, {\color{gray} 2}, {\color{gray} 3}, {\color{gray} 4}, {\bf 5}, {\color{gray} 6}, {\bf 7}, {\color{gray} 8}, {\color{gray} 9}, {\bf 10}, {\color{gray} 11}, {\bf 12}, {\color{gray} 13}, {\bf 14}, {\bf 15}, {\color{gray} 16}, {\bf 17}, {\color{gray} 18}, {\bf 19}, {\bf 20}, {\bf 21}, {\bf 22}, {\color{gray} 23}, {\bf 24}, {\bf 25}, \cdots$\\}

~

\end{center}
\end{frame}

%------------------------------------------------

\begin{frame}
\begin{center}
\frametitle{Example}

Player 2: $23$\\

~\\

M = $\lbrace 10, 5, 7, 23 \rbrace$\\

~\\

~\\

~\\

~\\

~\\

~\\

{\small ${\bf 0}, {\color{gray} 1}, {\color{gray} 2}, {\color{gray} 3}, {\color{gray} 4}, {\bf 5}, {\color{gray} 6}, {\bf 7}, {\color{gray} 8}, {\color{gray} 9}, {\bf 10}, {\color{gray} 11}, {\bf 12}, {\color{gray} 13}, {\bf 14}, {\bf 15}, {\color{gray} 16}, {\bf 17}, {\color{gray} 18}, {\bf 19}, {\bf 20}, {\bf 21}, {\bf 22}, {\bf 23}, {\bf 24}, {\bf 25}, \cdots$\\}

~

\end{center}
\end{frame}

%------------------------------------------------

\begin{frame}
\begin{center}
\frametitle{Example}

Player 1: $6$\\

~\\

M = $\lbrace 10, 5, 7, 23, 6 \rbrace$\\

~\\

~\\

~\\

~\\

~\\

~\\

{\small ${\bf 0}, {\color{gray} 1}, {\color{gray} 2}, {\color{gray} 3}, {\color{gray} 4}, {\bf 5}, {\bf 6}, {\bf 7}, {\color{gray} 8}, {\color{gray} 9}, {\bf 10}, {\bf 11}, {\bf 12}, {\bf 13}, {\bf 14}, {\bf 15}, {\bf 16}, {\bf 17}, {\bf 18}, {\bf 19}, {\bf 20}, {\bf 21}, {\bf 22}, {\bf 23}, {\bf 24}, {\bf 25}, \cdots$\\}

~

\end{center}
\end{frame}

%------------------------------------------------

\begin{frame}
\begin{center}
\frametitle{Example}

Player 2: $4$\\

~\\

M = $\lbrace 10, 5, 7, 23, 6, 4 \rbrace$\\

~\\

~\\

~\\

~\\

~\\

~\\

{\small ${\bf 0}, {\color{gray} 1}, {\color{gray} 2}, {\color{gray} 3}, {\bf 4}, {\bf 5}, {\bf 6}, {\bf 7}, {\bf 8}, {\bf 9}, {\bf 10}, {\bf 11}, {\bf 12}, {\bf 13}, {\bf 14}, {\bf 15}, {\bf 16}, {\bf 17}, {\bf 18}, {\bf 19}, {\bf 20}, {\bf 21}, {\bf 22}, {\bf 23}, {\bf 24}, {\bf 25}, \cdots$\\}

~

\end{center}
\end{frame}

%------------------------------------------------

\begin{frame}
\begin{center}
\frametitle{Example}

Player 1: $3$\\

~\\

M = $\lbrace 10, 5, 7, 23, 6, 4, 3 \rbrace$\\

~\\

~\\

~\\

~\\

~\\

~\\

{\small ${\bf 0}, {\color{gray} 1}, {\color{gray} 2}, {\bf 3}, {\bf 4}, {\bf 5}, {\bf 6}, {\bf 7}, {\bf 8}, {\bf 9}, {\bf 10}, {\bf 11}, {\bf 12}, {\bf 13}, {\bf 14}, {\bf 15}, {\bf 16}, {\bf 17}, {\bf 18}, {\bf 19}, {\bf 20}, {\bf 21}, {\bf 22}, {\bf 23}, {\bf 24}, {\bf 25}, \cdots$\\}

~

\end{center}
\end{frame}

%------------------------------------------------

\begin{frame}
\begin{center}
\frametitle{Example}

Player 2: $2$\\

~\\

M = $\lbrace 10, 5, 7, 23, 6, 4, 3, 2 \rbrace$\\

~\\

~\\

~\\

~\\

~\\

~\\

{\small ${\bf 0}, {\color{red} 1}, {\bf 2}, {\bf 3}, {\bf 4}, {\bf 5}, {\bf 6}, {\bf 7}, {\bf 8}, {\bf 9}, {\bf 10}, {\bf 11}, {\bf 12}, {\bf 13}, {\bf 14}, {\bf 15}, {\bf 16}, {\bf 17}, {\bf 18}, {\bf 19}, {\bf 20}, {\bf 21}, {\bf 22}, {\bf 23}, {\bf 24}, {\bf 25}, \cdots$\\}

Player 2 Wins!

\end{center}
\end{frame}

%------------------------------------------------
\section{Numerical Semigroup}
%------------------------------------------------

\begin{frame}
\begin{center}
\frametitle{Numerical Semigroups}

A subset $S$ of $\mathbb{N}$ closed under addition, containing zero, and having a largest integer not in $S$\\

~\\

OR\\

~\\

The set $S$ of all non-negative coefficient linear combinations of a set of generators whose greatest common divisor is $1$

\end{center}
\end{frame}

%------------------------------------------------

\begin{frame}
\begin{center}
\frametitle{Example}

Player 1: $7$\\

~\\

M = $\lbrace 10, 5, 7 \rbrace$\\

~\\

~\\

~\\

~\\

~\\

~\\

{\small ${\bf 0}, {\color{gray} 1}, {\color{gray} 2}, {\color{gray} 3}, {\color{gray} 4}, {\bf 5}, {\color{gray} 6}, {\bf 7}, {\color{gray} 8}, {\color{gray} 9}, {\bf 10}, {\color{gray} 11}, {\bf 12}, {\color{gray} 13}, {\bf 14}, {\bf 15}, {\color{gray} 16}, {\bf 17}, {\color{gray} 18}, {\bf 19}, {\bf 20}, {\bf 21}, {\bf 22}, {\color{gray} 23}, {\bf 24}, {\bf 25}, \cdots$\\}

~

\end{center}
\end{frame}

%------------------------------------------------

\begin{frame}
\begin{center}
\frametitle{Example}

~\\

~\\

S = $\langle 5, 7 \rangle$\\

~\\

~\\

~\\

~\\

~\\

~\\

{\small ${\bf 0}, {\color{gray} 1}, {\color{gray} 2}, {\color{gray} 3}, {\color{gray} 4}, {\bf 5}, {\color{gray} 6}, {\bf 7}, {\color{gray} 8}, {\color{gray} 9}, {\bf 10}, {\color{gray} 11}, {\bf 12}, {\color{gray} 13}, {\bf 14}, {\bf 15}, {\color{gray} 16}, {\bf 17}, {\color{gray} 18}, {\bf 19}, {\bf 20}, {\bf 21}, {\bf 22}, {\color{gray} 23}, {\bf 24}, {\bf 25}, \cdots$\\}

~

\end{center}
\end{frame}

%------------------------------------------------
\section{Definitions}
%------------------------------------------------

\begin{frame}
\begin{center}
\frametitle{Definitions}

Numerical Semigroups and Sylver Coinage have

very different terminology for the same properties

\end{center}
\end{frame}

%------------------------------------------------

\begin{frame}
\begin{center}
\frametitle{Moves/Generators}

~\\

M = $\lbrace 10, 5, 7 \rbrace$ Moves\\

S = $\langle 5, 7 \rangle$ Generators\\

~\\
{\flushleft
M - Position, a set of moves\\

S - Numerical Semigroup, a set of linear combinations\\
}
~\\

~\\

~\\

{\small ${\bf 0}, {\color{gray} 1}, {\color{gray} 2}, {\color{gray} 3}, {\color{gray} 4}, {\bf 5}, {\color{gray} 6}, {\bf 7}, {\color{gray} 8}, {\color{gray} 9}, {\bf 10}, {\color{gray} 11}, {\bf 12}, {\color{gray} 13}, {\bf 14}, {\bf 15}, {\color{gray} 16}, {\bf 17}, {\color{gray} 18}, {\bf 19}, {\bf 20}, {\bf 21}, {\bf 22}, {\color{gray} 23}, {\bf 24}, {\bf 25}, \cdots$\\}

~

\end{center}
\end{frame}

%------------------------------------------------

\begin{frame}
\begin{center}
\frametitle{$\mathscr{N}$ and $\mathscr{P}$}

~\\

~\\

~\\

~\\
{\flushleft
$\mathscr{N}$ - a position $M$ where the next player can win\\

$\mathscr{P}$ - all other positions, previous to $\mathscr{N}$\\
}
~\\

~\\

~\\

{\small ~\\}

~

\end{center}
\end{frame}

%------------------------------------------------

\begin{frame}
\begin{center}
\frametitle{g.c.d.}

~\\

M = $\lbrace 10, 5, 7 \rbrace$\\

S = $\langle 5, 7 \rangle$\\

~\\
{\flushleft
g - $g.c.d. \left( M \right)$\\

Note: $g = 1$ for Numerical Semigroup\\
}
~\\

~\\

~\\

{\small ${\bf 0}, {\color{gray} 1}, {\color{gray} 2}, {\color{gray} 3}, {\color{gray} 4}, {\bf 5}, {\color{gray} 6}, {\bf 7}, {\color{gray} 8}, {\color{gray} 9}, {\bf 10}, {\color{gray} 11}, {\bf 12}, {\color{gray} 13}, {\bf 14}, {\bf 15}, {\color{gray} 16}, {\bf 17}, {\color{gray} 18}, {\bf 19}, {\bf 20}, {\bf 21}, {\bf 22}, {\color{gray} 23}, {\bf 24}, {\bf 25}, \cdots$\\}

~

\end{center}
\end{frame}

%------------------------------------------------

\begin{frame}
\begin{center}
\frametitle{Frobenius Number}

~\\

M = $\lbrace 10, 5, 7 \rbrace$\\

S = $\langle 5, 7 \rangle$\\

~\\
{\flushleft
$t \left( M \right)$ - If $g.c.d \left( M \right) = 1$, $t \left( M \right)$ is the greatest legal move\\

$F \left( S \right)$ - greatest number not in $S$\\
}

~\\

~\\

~\\

{\small ${\bf 0}, {\color{gray} 1}, {\color{gray} 2}, {\color{gray} 3}, {\color{gray} 4}, {\bf 5}, {\color{gray} 6}, {\bf 7}, {\color{gray} 8}, {\color{gray} 9}, {\bf 10}, {\color{gray} 11}, {\bf 12}, {\color{gray} 13}, {\bf 14}, {\bf 15}, {\color{gray} 16}, {\bf 17}, {\color{gray} 18}, {\bf 19}, {\bf 20}, {\bf 21}, {\bf 22}, {\color{red} 23}, {\bf 24}, {\bf 25}, \cdots$\\}

~

\end{center}
\end{frame}

%------------------------------------------------

\begin{frame}
\begin{center}
\frametitle{End}

~\\

M = $\lbrace 10, 5, 7, 23 \rbrace$\\

S = $\langle 5, 7, 23 \rangle$\\

~\\
{\flushleft
End - all moves $x$ that do not eliminate any other move\\

~\\
}
~\\

~\\

~\\

{\small ${\bf 0}, {\color{gray} 1}, {\color{gray} 2}, {\color{gray} 3}, {\color{gray} 4}, {\bf 5}, {\color{gray} 6}, {\bf 7}, {\color{gray} 8}, {\color{gray} 9}, {\bf 10}, {\color{gray} 11}, {\bf 12}, {\color{gray} 13}, {\bf 14}, {\bf 15}, {\color{red} 16}, {\bf 17}, {\color{red} 18}, {\bf 19}, {\bf 20}, {\bf 21}, {\bf 22}, {\bf 23}, {\bf 24}, {\bf 25}, \cdots$\\}

~

\end{center}
\end{frame}

%------------------------------------------------

\begin{frame}
\begin{center}
\frametitle{Pesudo-Frobenius Number}

~\\

M = $\lbrace 10, 5, 7, 23 \rbrace$\\

S = $\langle 5, 7, 23 \rangle$\\

~\\
{\flushleft
~\\

$PF \left( S \right)$ - set of all integers $x$ such that $x \notin S$ and $x + s \in S$\\
}
~\\

~\\

~\\

{\small ${\bf 0}, {\color{gray} 1}, {\color{gray} 2}, {\color{gray} 3}, {\color{gray} 4}, {\bf 5}, {\color{gray} 6}, {\bf 7}, {\color{gray} 8}, {\color{gray} 9}, {\bf 10}, {\color{gray} 11}, {\bf 12}, {\color{gray} 13}, {\bf 14}, {\bf 15}, {\color{red} 16}, {\bf 17}, {\color{red} 18}, {\bf 19}, {\bf 20}, {\bf 21}, {\bf 22}, {\bf 23}, {\bf 24}, {\bf 25}, \cdots$\\}

~

\end{center}
\end{frame}

%------------------------------------------------

\begin{frame}
\begin{center}
\frametitle{Ender}

~\\

M = $\lbrace 10, 5, 7, 23, 16 \rbrace$\\

S = $\langle 5, 7, 16 \rangle$\\

~\\
{\flushleft
Ender - A position $M$ where $t \left( M \right)$ is the only end\\

$S$ is irreducible, i.e. pesudo-symmetric or symmetric\\

$PF \left( S \right) = \lbrace F \left( S \right) / 2,  F \left( S \right) \rbrace$ or $\lbrace F \left( S \right) \rbrace$\\

Enders are $\mathscr{N}$, allow for strategy stealing\\
}

~\\

{\small ${\bf 0}, {\color{gray} 1}, {\color{gray} 2}, {\color{gray} 3}, {\color{gray} 4}, {\bf 5}, {\color{gray} 6}, {\bf 7}, {\color{gray} 8}, {\color{gray} 9}, {\bf 10}, {\color{gray} 11}, {\bf 12}, {\color{gray} 13}, {\bf 14}, {\bf 15}, {\bf 16}, {\bf 17}, {\color{red} 18}, {\bf 19}, {\bf 20}, {\bf 21}, {\bf 22}, {\bf 23}, {\bf 24}, {\bf 25}, \cdots$\\}

~

\end{center}
\end{frame}

%------------------------------------------------

\begin{frame}
\begin{center}
\frametitle{Ender}

~\\

M = $\lbrace 10, 5, 7, 23, 16 \rbrace$\\

S = $\langle 5, 7, 16 \rangle$\\

~\\
{\flushleft
Ender - A position $M$ where $t \left( M \right)$ is the only end\\

$S$ is irreducible, i.e. pesudo-symmetric or symmetric\\

$PF \left( S \right) = \lbrace F \left( S \right) / 2,  F \left( S \right) \rbrace$ or $\lbrace F \left( S \right) \rbrace$\\

Enders are $\mathscr{N}$, allow for strategy stealing\\
}

~\\

{\small ${\bf 0}, {\color{gray} 1}, {\color{gray} 2}, {\color{gray} 3}, {\color{gray} 4}, {\bf 5}, {\color{gray} 6}, {\bf 7}, {\color{gray} 8}, {\color{blue} 9}, {\bf 10}, {\color{gray} 11}, {\bf 12}, {\color{gray} 13}, {\bf 14}, {\bf 15}, {\bf 16}, {\bf 17}, {\color{red} 18}, {\bf 19}, {\bf 20}, {\bf 21}, {\bf 22}, {\bf 23}, {\bf 24}, {\bf 25}, \cdots$\\}

~

\end{center}
\end{frame}

%------------------------------------------------

\begin{frame}
\begin{center}
\frametitle{Quiet Ender}

~\\

M = $\lbrace 10, 5, 7 \rbrace$\\

S = $\langle 5, 7 \rangle$\\

~\\
{\flushleft
Quiet Ender - Ender and every move eliminates $t \left( M \right)$ without multiples\\

$PF \left( S \right) = \lbrace F \left( S \right) \rbrace$, $S$ is symmetric\\

~\\
}

~\\

~\\

{\small ${\bf 0}, {\color{gray} 1}, {\color{gray} 2}, {\color{gray} 3}, {\color{gray} 4}, {\bf 5}, {\color{gray} 6}, {\bf 7}, {\color{gray} 8}, {\color{gray} 9}, {\bf 10}, {\color{gray} 11}, {\bf 12}, {\color{gray} 13}, {\bf 14}, {\bf 15}, {\color{gray} 16}, {\bf 17}, {\color{gray} 18}, {\bf 19}, {\bf 20}, {\bf 21}, {\bf 22}, {\color{red} 23}, {\bf 24}, {\bf 25}, \cdots$\\}

~

\end{center}
\end{frame}

%------------------------------------------------

\begin{frame}
\begin{center}
\frametitle{Quiet Ender}

~\\

M = $\lbrace 10, 5, 7 \rbrace$\\

S = $\langle 5, 7 \rangle$\\

~\\
{\flushleft
Quiet Ender - Ender and every move eliminates $t \left( M \right)$ without multiples\\

$PF \left( S \right) = \lbrace F \left( S \right) \rbrace$, $S$ is {\bf symmetric}\\

$M$ displays Strong {\bf Antisymmetery} Principle\\
}

~\\

~\\

{\small ${\bf 0}, {\color{gray} 1}, {\color{gray} 2}, {\color{gray} 3}, {\color{gray} 4}, {\bf 5}, {\color{gray} 6}, {\bf 7}, {\color{gray} 8}, {\color{gray} 9}, {\bf 10}, {\color{gray} 11}, {\bf 12}, {\color{gray} 13}, {\bf 14}, {\bf 15}, {\color{gray} 16}, {\bf 17}, {\color{gray} 18}, {\bf 19}, {\bf 20}, {\bf 21}, {\bf 22}, {\color{red} 23}, {\bf 24}, {\bf 25}, \cdots$\\}

~

\end{center}
\end{frame}

%------------------------------------------------

\begin{frame}
\begin{center}
\frametitle{Quiet Ender}

~\\

M = $\lbrace 10, 5, 7 \rbrace$\\

S = $\langle 5, 7 \rangle$\\

~\\
{\flushleft
Quiet Ender - Ender and every move can eliminate $t \left( M \right)$ without multiples\\

$PF \left( S \right) = \lbrace F \left( S \right) \rbrace$, $S$ is symmetric\\

$M$ displays Strong Antisymmetery Principle\\
}

~\\

~\\

{\small ${\bf 0}, {\color{gray} 1}, {\color{gray} 2}, {\color{gray} 3}, {\color{gray} 4}, {\bf 5}, {\color{gray} 6}, {\bf \color{blue} 7}, {\color{gray} 8}, {\color{gray} 9}, {\bf 10}, {\color{gray} 11}, {\bf 12}, {\color{gray} 13}, {\bf 14}, {\bf 15}, {\color{blue} 16}, {\bf 17}, {\color{gray} 18}, {\bf 19}, {\bf 20}, {\bf 21}, {\bf 22}, {\color{red} 23}, {\bf 24}, {\bf 25}, \cdots$\\}

~

\end{center}
\end{frame}

%------------------------------------------------
\section{Strategies}
%------------------------------------------------

\begin{frame}
\begin{center}
\frametitle{Strategies}

Some strategies are known\\

~\\

Research in Numerical Semigroups provides

other strategies for winning

\end{center}
\end{frame}

%------------------------------------------------

\begin{frame}
\begin{center}
\frametitle{Known Strategies}

~\\

\begin{enumerate}

\item If $p \geq 5$, then $\lbrace p \rbrace$ is $\mathscr{P}$

\item If $p \geq 5$, then $\lbrace a p \rbrace$ loses to $p$

\item If $M$ is nontrivial and $g \geq 5$ is prime, then $g$ wins

\item If $a, b \in \mathbb{Z}^+$, then $\lbrace 2^a 3^b \rbrace$ is generally unknown

\end{enumerate}

\end{center}
\end{frame}

%------------------------------------------------

\begin{frame}
\begin{center}
\frametitle{Enclosure}

~\\

M = $\lbrace 10, 5, 7, 23, 16, 9 \rbrace$\\

S = $\langle 5, 7, 9, 16 \rangle$\\

~\\
{\flushleft
$E \left( M \right)$ - Position obtained by imposing antisymmetry on $M$\\

$E \left( M \right) = \lbrace 10, 5, 7, 23, 16, 9, 11 \rbrace$\\
}

~\\

~\\

~\\

{\small ${\bf 0}, {\color{gray} 1}, {\color{gray} 2}, {\color{gray} 3}, {\color{gray} 4}, {\bf 5}, {\color{gray} 6}, {\bf 7}, {\color{gray} 8}, {\bf 9}, {\bf 10}, {\color{blue} 11}, {\bf 12}, {\color{red} 13}, {\bf 14}, {\bf 15}, {\bf 16}, {\bf 17}, {\bf 18}, {\bf 19}, {\bf 20}, {\bf 21}, {\bf 22}, {\bf 23}, {\bf 24}, {\bf 25}, \cdots$\\}

~

\end{center}
\end{frame}

%------------------------------------------------

\begin{frame}
\begin{center}
\frametitle{Strategies}

~\\

\begin{enumerate}

\item Completing $E \left( M \right)$ results in $\mathscr{N}$

\item $F \left( S \right)$ is not always the winning move with an ender

\end{enumerate}

\end{center}
\end{frame}

%------------------------------------------------

\begin{frame}
\begin{center}
\frametitle{Playing $F \left( S \right)$}

~\\

M = $\lbrace 8, 7, 4 \rbrace$\\

S = $\langle 4, 7 \rangle$\\

~\\
{\flushleft
Picking $F \left( S \right) = 17$ will result in loss\\

~\\
}

~\\

~\\

~\\

{\small ${\bf 0}, {\color{gray} 1}, {\color{gray} 2}, {\color{gray} 3}, {\bf 4}, {\color{gray} 5}, {\color{gray} 6}, {\bf 7}, {\bf 8}, {\color{gray} 9}, {\color{gray} 10}, {\bf 11}, {\bf 12}, {\color{gray} 13}, {\bf 14}, {\bf 15}, {\bf 16}, {\color{blue} 17}, {\bf 18}, {\bf 19}, {\bf 20}, {\bf 21}, {\bf 22}, {\bf 23}, {\bf 24}, {\bf 25}, \cdots$\\}

~

\end{center}
\end{frame}

%------------------------------------------------

\begin{frame}
\begin{center}
\frametitle{Playing $F \left( S \right)$}

~\\

M = $\lbrace 8, 7, 4, 17 \rbrace$\\

S = $\langle 4, 7, 17 \rangle$\\

~\\
{\flushleft
Picking $F \left( S \right) = 17$ will result in loss\\

~\\
}

~\\

~\\

~\\

{\small ${\bf 0}, {\color{gray} 1}, {\color{gray} 2}, {\color{gray} 3}, {\bf 4}, {\color{gray} 5}, {\color{gray} 6}, {\bf 7}, {\bf 8}, {\color{gray} 9}, {\color{gray} 10}, {\bf 11}, {\bf 12}, {\color{blue} 13}, {\bf 14}, {\bf 15}, {\bf 16}, {\bf 17}, {\bf 18}, {\bf 19}, {\bf 20}, {\bf 21}, {\bf 22}, {\bf 23}, {\bf 24}, {\bf 25}, \cdots$\\}

~

\end{center}
\end{frame}

%------------------------------------------------

\begin{frame}
\begin{center}
\frametitle{Playing $F \left( S \right)$}

~\\

M = $\lbrace 8, 7, 4, 17, 13 \rbrace$\\

S = $\langle 4, 7, 13, 17 \rangle$\\

~\\
{\flushleft
Picking $F \left( S \right) = 17$ will result in loss\\

Player 2 picks $13$\\

Player 1 cannot make a winning move\\
}


~\\

~\\

{\small ${\bf 0}, {\color{gray} 1}, {\color{gray} 2}, {\color{gray} 3}, {\bf 4}, {\color{gray} 5}, {\color{gray} 6}, {\bf 7}, {\bf 8}, {\color{gray} 9}, {\color{blue} 10}, {\bf 11}, {\bf 12}, {\bf 13}, {\bf 14}, {\bf 15}, {\bf 16}, {\bf 17}, {\bf 18}, {\bf 19}, {\bf 20}, {\bf 21}, {\bf 22}, {\bf 23}, {\bf 24}, {\bf 25}, \cdots$\\}

~

\end{center}
\end{frame}

%------------------------------------------------

\begin{frame}
\begin{center}
\frametitle{Strategies}

~\\

\begin{enumerate}

\item Completing $E \left( M \right)$ results in $\mathscr{N}$

\item $F \left( S \right)$ is not always the winning move with an ender

\item Let $m$ is the smallest move and $e$ is the number of moves

\item If $S = \langle p, q \rangle$, then $S = \langle p, q, n q - p \rangle$ will be symmetric if $n \vert p$

\end{enumerate}

\end{center}
\end{frame}

%------------------------------------------------

\begin{frame}
\begin{center}
\frametitle{Strategies}

~\\

\begin{enumerate}

\item Completing $E \left( M \right)$ results in $\mathscr{N}$

\item $F \left( S \right)$ is not always the winning move with an ender

\item If $S = \langle p, q \rangle$, then $S = \langle p, q, n q - p \rangle$ will be symmetric if $n \vert p$

\item Let $m$ is the smallest move and $e$ is the number of independent moves

\begin{enumerate}

\item If $2 \leq e \leq m - 1$, then there exists a irreducible numerical semigroup with these values of $m$ and $e$

\item Always need to be aware of how close semigroup is to irreducible

\end{enumerate}

\end{enumerate}

\end{center}
\end{frame}

%------------------------------------------------
\section{Future Questions}
%------------------------------------------------

\begin{frame}
\begin{center}
\frametitle{Future Questions}

Some of Richard Guy\rq{}s questions:

\begin{enumerate}

\item Is there an effective technique for computing the status of any $M$?

\item Is there an effective technique for producing good replies?

\item What is the status of position $\lbrace n \rbrace$ for $n$ of the form $2^a 3^b$?

\item What is the status of $\lbrace 16 \rbrace$?

\item What is the status of $\lbrace 18 \rbrace$?

\item Is M a $\mathscr{P}$-position whenever 2$M$ is a $\mathscr{P}$-position?

\item If the game is played \lq\lq{}between intelligent players\rq\rq{} is it always the case that the first person to make the game bounded is the loser?

\end{enumerate}

\end{center}
\end{frame}
%------------------------------------------------

\begin{frame}[noframenumbering]
\titlepage % Print the title page
\end{frame}

%------------------------------------------------

%----------------------------------------------------------------------------------------

\end{document} 