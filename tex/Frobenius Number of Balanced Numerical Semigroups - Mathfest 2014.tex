%%%%%%%%%%%%%%%%%%%%%%%%%%%%%%%%%%%%%%%%%
% Beamer Presentation
% LaTeX Template
% Version 1.0 (10/11/12)
%
% This template has been downloaded from:
% http://www.LaTeXTemplates.com
%
% License:
% CC BY-NC-SA 3.0 (http://creativecommons.org/licenses/by-nc-sa/3.0/)
%
%%%%%%%%%%%%%%%%%%%%%%%%%%%%%%%%%%%%%%%%%

%----------------------------------------------------------------------------------------
%	PACKAGES AND THEMES
%----------------------------------------------------------------------------------------

\documentclass{beamer}

\mode<presentation> {

\usetheme{CambridgeUS}

\usecolortheme{dolphin}
}

\usepackage{graphicx} % Allows including images
\usepackage{booktabs} % Allows the use of \toprule, \midrule and \bottomrule in tables

%----------------------------------------------------------------------------------------
%	TITLE PAGE
%----------------------------------------------------------------------------------------

\title[Balanced Numerical Semigroups]{The Frobenius Number of\\ Balanced Numerical Semigroups} % The short title appears at the bottom of every slide, the full title is only on the title page

\author{Jeremy L Thompson} % Your name
\institute[USAFA] % Your institution as it will appear on the bottom of every slide, may be shorthand to save space
{United States Air Force Academy \\ % Your institution for the title page
\medskip
\textit{jeremy.thompson@usafa.edu} % Your email address
}
\date{\today} % Date, can be changed to a custom date

\begin{document}

\begin{frame}
\titlepage % Print the title page as the first slide
\end{frame}

%------------------------------------------------

\begin{frame}
\begin{center}
\frametitle{Overview}

We will discuss the algebraic structure known as a numerical semigroup and definitions related to them.\\

~\\

We will investigate the Frobenius number of balanced numerical semigroups.

\end{center}
\end{frame}
 
%------------------------------------------------

\begin{frame}
\frametitle{Overview} % Table of contents slide, comment this block out to remove it
\tableofcontents % Throughout your presentation, if you choose to use \section{} and \subsection{} commands, these will automatically be printed on this slide as an overview of your presentation
\end{frame}

%----------------------------------------------------------------------------------------
%	PRESENTATION SLIDES
%----------------------------------------------------------------------------------------

%------------------------------------------------
\section{Background}
%------------------------------------------------

\begin{frame}
\begin{center}
\frametitle{Coin Exchange Problem}

Given $5$ and $7$ cent coins, what is the largest amount of change that cannot be created?\\

~\\

~\\

\only<-1 | handout : 0>{\small {\color{gray} $0, 1, 2, 3, 4, 5, 6, 7, 8, 9, 10, 11, 12, 13, 14, 15, 16, 17, 18, 19, 20, 21, 22, 23, 24, 25 \cdots$}}

\only<2- | handout : 0>{\small ${\bf 0}, {\color{gray} 1}, {\color{gray} 2}, {\color{gray} 3}, {\color{gray} 4}, {\bf 5}, {\color{gray} 6}, {\bf 7}, {\color{gray} 8}, {\color{gray} 9}, {\bf 10}, {\color{gray} 11}, {\bf 12}, {\color{gray} 13}, {\bf 14}, {\bf 15}, {\color{gray} 16}, {\bf 17}, {\color{gray} 18}, {\bf 19}, {\bf 20}, {\bf 21}, {\bf 22}, {\bf \color{red} 23}, {\bf 24}, {\bf 25}, \cdots$\\}

~\\

\uncover<3- | handout : 0>{$23 = 7 \cdot 5 - 7 - 5$}

\end{center}
\end{frame}

%------------------------------------------------

\begin{frame}
\begin{center}
\frametitle{Numerical Semigroups}

A subset $S$ of $\mathbb{N}$ closed under addition, containing zero, and having a largest integer not in $S$\\

~\\

~\\

~\\

~\\

~\\

~\\

~\\

{\small ${\bf 0}, {\color{gray} 1}, {\color{gray} 2}, {\color{gray} 3}, {\color{gray} 4}, {\color{gray} 5}, {\color{gray} 6}, {\color{gray} 7}, {\bf 8}, {\bf 9}, {\color{gray} 10}, {\bf 11}, {\bf 12}, {\color{gray} 13}, {\color{gray} 14}, {\color{gray} 15}, {\bf 16}, {\bf 17}, {\bf 18}, {\bf 19}, {\bf 20}, {\bf 21}, {\bf 22}, {\bf 23}, \cdots$}

\end{center}
\end{frame}
 
%------------------------------------------------

\begin{frame}
\begin{center}
\frametitle{Numerical Semigroups}

Let $a _ 1, a _ 2, ... , a _ n \in \mathbb{N}$ such that $gcd \left( a _ 1, a _ 2, ... , a _ n \right) = 1$.\\

~\\

$S = \langle a _ 1, a _ 2, ... , a _ n \rangle = \{ a _ 1 t _ 1 + ... + a _ n t _ n \vert t _ i \in \mathbb{N} \}$\\
$x \in S = \left(t_1, t_2, \cdots, t_n\right)$\\

~\\

\uncover<2- | handout : 0>{$a _ 1, a _ 2, ... , a _ n$ are called generators}

~\\

~\\

~\\

{\small ${\bf 0}, {\color{gray} 1}, {\color{gray} 2}, {\color{gray} 3}, {\color{gray} 4}, {\color{gray} 5}, {\color{gray} 6}, {\color{gray} 7}, {\bf 8}, {\bf 9}, {\color{gray} 10}, {\bf 11}, {\bf 12}, {\color{gray} 13}, {\color{gray} 14}, {\color{gray} 15}, {\bf 16}, {\bf 17}, {\bf 18}, {\bf 19}, {\bf 20}, {\bf 21}, {\bf 22}, {\bf 23}, \cdots$}

\end{center}
\end{frame}
 
%------------------------------------------------

\begin{frame}
\begin{center}
\frametitle{Frobenius Number}

\uncover<1- | handout : 0>{Frobenius Number: largest integer not in $S$\\~\\

$S = \langle 8, 9, 11, 12 \rangle$\\}

~\\

$F(S) = 15$

~\\

~\\

~\\

~\\

{\small ${\bf 0}, {\color{gray} 1}, {\color{gray} 2}, {\color{gray} 3}, {\color{gray} 4}, {\color{gray} 5}, {\color{gray} 6}, {\color{gray} 7}, {\bf 8}, {\bf 9}, {\color{gray} 10}, {\bf 11}, {\bf 12}, {\color{gray} 13}, {\color{gray} 14}, {\bf \color{red} 15}, {\bf 16}, {\bf 17}, {\bf 18}, {\bf 19}, {\bf 20}, {\bf 21}, {\bf 22}, {\bf 23}, \cdots$}

\end{center}
\end{frame}

%------------------------------------------------

\begin{frame}
\begin{center}
\frametitle{Known Results}

"It would be nice to have a finite set of formulas which covers all possible cases for computing $F(S)$. Unfortunately, no such collection of formulas has ben found and probably does not exist at all."\\

~\\

- {\textit 'The exact solutions to the Frobenius problem with three variables'} by Mihaly Hujter and Bela Vizvri; J. Ramanujan Math. Soc. 2 $(1987)$

\end{center}
\end{frame}

%------------------------------------------------

\begin{frame}
\begin{center}
\frametitle{Known Results}

\small{

\begin{itemize}

\item $2$ Generators - $S = \langle a, b \rangle$ (Sylvester 1884)\\

$F(S) = a \cdot b - a - b$

\item $3+$ Generators - $S = \langle a_1, a_2, \cdots, a_n \rangle$\\

$F(S) = ???$

\uncover<2- | handout : 0>{

\item Arithmetic sequence $S = \langle a, ma+d, ma + 2d, \cdots, ma + kd \rangle$ (Lewin 1975)\\

$F(S) = m \cdot a \cdot \left( 1 + \lfloor \frac{a - 2}{k} \rfloor \right) + \left( d - 1 \right)\left(a - 1 \right) - 1$

\item $S = <a_1, \dots , a_n>$ (not necessarily in ascending order) and let $d = gcd\{a_1, \dots , a_{n - 1}\}$ (Brauer and Schockley 1962)

$F(S) = d \cdot F(\langle \frac{a_1}{d}, \dots , \frac{a_{n-1}}{d}, a_n \rangle) + \left( d - 1 \right) a_n$

\item  $S = <a_1,a_2,a_3>$ and $a_i | (a_1+a_2+a_3)$ for some $i=2,3$ (Brauer and Schockley 1962)\\

$F(S) = - a_1 + max_{i = 2, 3}\{a_i \cdot \lfloor \frac{a_1 \cdot a_{5 - i}}{a_2 + a_3} \rfloor\}$
}

\end{itemize}

}

\end{center}
\end{frame}

%------------------------------------------------

\begin{frame}
\begin{center}
\frametitle{Apery Set Relative to $a_1$}

Apery Set relative to $a_1$:\\

~\\

All elements in $S$ such that\\

$x \in S$ and $x - a_1 \notin S$\\

~\\

\uncover<2- | handout : 0>{or: all elements in $S$ that can only be made without $a_1$\\

~\\

~\\

Note: $Max(Ap(S)) - a_1 = F(S)$}

\end{center}
\end{frame}

%------------------------------------------------

\begin{frame}
\begin{center}
\frametitle{Apery Set Relative to $a_1$}

Apery Set Relative to $a_1$ ($8$):\\

$x \in S$ and $x - 8 \notin S$\\~\\

$S = \langle 8, 9, 11, 12 \rangle$\\

~\\

\uncover<3- | handout : 0>{$Ap(S) = \{ 0, 9, 11, 12, 18, 21, 22, 23 \}$\\}

~\\

\uncover<4- | handout : 0>{Note: $Max(Ap(S)) - 8 = 23 - 8 = 15 = F(S)$}\\

~\\

{\small
\only<1-1 | handout : 0>{\small ${\bf 0}, {\color{gray} 1}, {\color{gray} 2}, {\color{gray} 3}, {\color{gray} 4}, {\color{gray} 5}, {\color{gray} 6}, {\color{gray} 7}, {\bf 8}, {\bf 9}, {\color{gray} 10}, {\bf 11}, {\bf 12}, {\color{gray} 13}, {\color{gray} 14}, {\color{gray} 15}, {\bf 16}, {\bf 17}, {\bf 18}, {\bf 19}, {\bf 20}, {\bf 21}, {\bf 22}, {\bf 23}, \cdots$}}

{\small
\only<2-2 | handout : 0>{\small ${\color{red} |} {\bf 0}, {\color{gray} 1}, {\color{gray} 2}, {\color{gray} 3}, {\color{gray} 4}, {\color{gray} 5}, {\color{gray} 6}, {\color{gray} 7} {\color{red} |} {\bf 8}, {\bf 9}, {\color{gray} 10}, {\bf 11}, {\bf 12}, {\color{gray} 13}, {\color{gray} 14}, {\color{gray} 15} {\color{red} |} {\bf 16}, {\bf 17}, {\bf 18}, {\bf 19}, {\bf 20}, {\bf 21}, {\bf 22}, {\bf 23} {\color{red} |} \cdots$}}

{\small
\only<3- | handout : 0>{\small $| {\bf \color{red} 0}, {\color{gray} 1}, {\color{gray} 2}, {\color{gray} 3}, {\color{gray} 4}, {\color{gray} 5}, {\color{gray} 6}, {\color{gray} 7} | {\bf 8}, {\bf \color{red} 9}, {\color{gray} 10}, {\bf \color{red} 11}, {\bf \color{red} 12}, {\color{gray} 13}, {\color{gray} 14}, {\bf \color{blue} 15} | {\bf 16}, {\bf 17}, {\bf \color{red} 18}, {\bf 19}, {\bf 20}, {\bf \color{red} 21}, {\bf \color{red} 22}, {\bf \color{red} 23} | \cdots$}}

\end{center}
\end{frame}

%------------------------------------------------

\begin{frame}
\begin{center}
\frametitle{Example}

$S = \langle 6, 8, 13, 15 \rangle$\\

~\\

\uncover<3- | handout : 0>{$F(S) = 17$}

~\\

\uncover<4- | handout : 0>{$Ap(S) = \{ 0, 8, 13, 15, 16, 23 \}$\\}

~\\

\only<1-1 | handout : 0>{
{\small ${\color{gray} 0}, {\color{gray} 1}, {\color{gray} 2}, {\color{gray} 3}, {\color{gray} 4}, {\color{gray} 5}, {\bf 6}, {\color{gray} 7}, {\bf 8}, {\color{gray} 9}, {\color{gray} 10}, {\color{gray} 11}, {\color{gray} 12}, {\bf 13}, {\color{gray} 14}, {\bf 15}, {\color{gray} 16}, {\color{gray} 17}, {\color{gray} 18}, {\color{gray} 19}, {\color{gray} 20}, {\color{gray} 21}, {\color{gray} 22}, {\color{gray} 23}, \cdots$}}

\only<2-2 | handout : 0>{
{\small ${\bf 0}, {\color{gray} 1}, {\color{gray} 2}, {\color{gray} 3}, {\color{gray} 4}, {\color{gray} 5}, {\bf 6}, {\color{gray} 7}, {\bf 8}, {\color{gray} 9}, {\color{gray} 10}, {\color{gray} 11}, {\bf 12}, {\bf 13}, {\bf 14}, {\bf 15}, {\bf 16}, {\color{gray} 17}, {\bf 18}, {\bf 19}, {\bf 20}, {\bf 21}, {\bf 22}, {\bf 23}, \cdots$}}

\only<3-3 | handout : 0>{
{\small ${\bf 0}, {\color{gray} 1}, {\color{gray} 2}, {\color{gray} 3}, {\color{gray} 4}, {\color{gray} 5}, {\bf 6}, {\color{gray} 7}, {\bf 8}, {\color{gray} 9}, {\color{gray} 10}, {\color{gray} 11}, {\bf 12}, {\bf 13}, {\bf 14}, {\bf 15}, {\bf 16}, {\bf \color{red} 17}, {\bf 18}, {\bf 19}, {\bf 20}, {\bf 21}, {\bf 22}, {\bf 23}, \cdots$}}

\only<4-4 | handout : 0>{
{\small $| {\bf \color{red} 0}, {\color{gray} 1}, {\color{gray} 2}, {\color{gray} 3}, {\color{gray} 4}, {\color{gray} 5} | {\bf 6}, {\color{gray} 7}, {\bf \color{red} 8}, {\color{gray} 9}, {\color{gray} 10}, {\color{gray} 11} | {\bf 12}, {\bf \color{red} 13}, {\bf 14}, {\bf \color{red} 15}, {\bf \color{red} 16}, {\bf \color{blue} 17} | {\bf 18}, {\bf 19}, {\bf 20}, {\bf 21}, {\bf 22}, {\bf \color{red} 23} | \cdots$}}

\end{center}
\end{frame}

%------------------------------------------------
\section{Balanced Numerical Semigroups}
%------------------------------------------------

\begin{frame}
\begin{center}
\frametitle{Balanced Numerical Semigroups}

$S = \langle p D, p D + n, q D - n, q D \rangle$\\

~\\

\uncover<2- | handout : 0>{$p, q, D, n \in \mathbb{N}$\\

~\\

$p < q$ and $gcd \left( p, q \right) = 1$\\

$gcd \left( D, n \right) = 1$\\

~\\

$D \ge 1$ and $1 \le n < \frac{p + q}{2} D$}

\end{center}
\end{frame}

%------------------------------------------------

\begin{frame}
\begin{center}
\frametitle{Example}

$S = \langle p D, p D + n, q D - n, q D \rangle$\\~\\

$p = 2, q = 3, D = 4,$ and $n = 1$\\

~\\

\uncover<2- | handout : 0>{$S = \langle p D, p D + n, q D - n, q D \rangle = \langle 8, 9, 11, 12 \rangle$}

~\\

\uncover<3- | handout : 0>{{\small ${\bf 0}, {\color{gray} 1}, {\color{gray} 2}, {\color{gray} 3}, {\color{gray} 4}, {\color{gray} 5}, {\color{gray} 6}, {\color{gray} 7}, {\bf 8}, {\bf 9}, {\color{gray} 10}, {\bf 11}, {\bf 12}, {\color{gray} 13}, {\color{gray} 14}, {\bf \color{red} 15}, {\bf 16}, {\bf 17}, {\bf 18}, {\bf 19}, {\bf 20}, {\bf 21}, {\bf 22}, {\bf 23}, \cdots$}\\

~\\

$F \left( S \right) = 15$}

\end{center}
\end{frame}

%------------------------------------------------
\section{Frobenius Number of Balanced Numerical Semigroups}
%------------------------------------------------

\begin{frame}
\begin{center}
\frametitle{Frobenius Number of Balanced Numerical Semigroups}

We use relationships among the generators to find elements of $Ap(S)$\\

~\\

If $x \in S$ and $x = t_1 a_1 + t_2 a_2 + t_3 a_3 + t_4 a_4$\\

~\\

We say $x = \left(t_1, t_2, t_3, t_4\right)$\\

~\\

$t_1, t_2, t_3, t_4 \ge 0$

\end{center}
\end{frame}

%------------------------------------------------

\begin{frame}
\begin{center}
\frametitle{Relations Among the Generators}

$S = \langle 8, 9, 11, 12 \rangle$

{\footnotesize
\begin{center}
\begin{tabular}{cl}
($A$) & $\left(3, 0, 0, 0\right) = \left(0, 0, 0, 2\right)$\\
 &\\
($B$) & $\left(1, 0, 0, 1\right) = \left(0, 1, 1, 0\right)$\\
 &\\
($C_0$) & $\left(2, 0, 1, 0\right) = \left(0, 3, 0, 0\right)$\\
 &\\
($D_0$) & $\left(3, 1, 0, 0\right) = \left(0, 0, 3, 0\right)$\\
 &\\
($C_1$) & $\left(1, 0, 2, 0\right) = \left(0, 2, 0, 1\right)$\\
 &\\
($D_1$) & $\left(2, 2, 0, 0\right) = \left(0, 0, 2, 1\right)$\\
\end{tabular}
\end{center}
}

\end{center}
\end{frame}

%------------------------------------------------

\begin{frame}
\begin{center}
\frametitle{Relations Among the Generators}

Note that

{\footnotesize
\begin{center}
\begin{tabular}{cl}
($A$) & $\left(3, 0, 0, 0\right)=\left(0, 0, 0, 2\right)$\\
\end{tabular}
\end{center}
}

~\\

implies\\

~\\

if $x = \left(0, 0, 0, a\right) \in Ap(S)$ then $a < 2$

\end{center}
\end{frame}

%------------------------------------------------

\begin{frame}
\begin{center}
\frametitle{Members of $Ap(S)$}

$S = \langle 8, 9, 11, 12 \rangle$

{\footnotesize
\begin{center}
\begin{tabular}{cl}
($A$) & $\left(3, 0, 0, 0\right) = \left(0, 0, 0, 2\right)$\\
 & if $x = \left(0, 0, 0, a\right) \in Ap(S)$ then $a < 2$\\
 &\\
($B$) & $\left(1, 0, 0, 1\right) = \left(0, 1, 1, 0\right)$\\
 &\\
($C_0$) & $\left(2, 0, 1, 0\right) = \left(0, 3, 0, 0\right)$\\
 & if $x = \left(0, c_0, 0, 0\right) \in Ap(S)$ then $c_0 < 3$\\
 &\\
($D_0$) & $\left(3, 1, 0, 0\right) = \left(0, 0, 3, 0\right)$\\
 & if $x = \left(0, 0, d_0, 0\right) \in Ap(S)$ then $d_0 < 3$\\
 &\\
($C_1$) & $\left(1, 0, 2, 0\right) = \left(0, 2, 0, 1\right)$\\
 & if $x = \left(0, c_1, 0, 1\right) \in Ap(S)$ then $c_1 < 2$\\
 &\\
($D_1$) & $\left(2, 2, 0, 0\right) = \left(0, 0, 2, 1\right)$\\
 & if $x = \left(0, 0, d_1, 1\right) \in Ap(S)$ then $d_1 < 2$\\
\end{tabular}
\end{center}
}

\end{center}
\end{frame}

%------------------------------------------------

\begin{frame}
\begin{center}
\frametitle{Members of $Ap(S)$}

$S = \langle 8, 9, 11, 12 \rangle$, $F(S) = 15$

\begin{center}
\begin{tabular}{cl}
($C_1$) & $\left(1, 0, 2, 0\right) = \left(0, 2, 0, 1\right)$\\
 & if $x = \left(0, c_1, 0, 1\right) \in Ap(S)$ then $c_1 < 2$\\
 &\\
($D_1$) & $\left(2, 2, 0, 0\right) = \left(0, 0, 2, 1\right)$\\
 & if $x = \left(0, 0, d_1, 1\right) \in Ap(S)$ then $d_1 < 2$\\
\end{tabular}
\end{center}

~\\

\uncover<2- | handout : 0>{$\left(0, 1, 0, 1\right) < \left(0, 0, 1, 1\right)$, so\\

~\\

$F(S) = \left(0, 0, 1, 1\right) - 8 = 11 + 12 - 8 = 15$}

\end{center}
\end{frame}

%------------------------------------------------

\begin{frame}
\begin{center}
\frametitle{Relations Among the Generators}

{\scriptsize
\begin{center}
\begin{tabular}{cl}
($A$) & $\left(q, 0, 0, 0\right)=\left(0, 0, 0, p\right)$\\
 &\\
($B$) & $\left(1, 0, 0, 1\right) = \left(0, 1, 1, 0\right)$\\
 &\\
($C_0$) & $\left(\frac{a p + a q + n}{p} + k, 0, \frac{p D}{p + q} - \frac{p k}{p + q} - a, 0\right)$\\
 & $ = \left(0, \frac{q D}{p + q} + \frac{p k}{p + q} + a, 0, 0\right)$\\
 &\\
($D_0$) & $\left(q - \frac{a p + a q + n}{p} - k + p, \frac{q D}{p + q} + \frac{p k}{p + q} + a - p, 0, 0\right)$\\
 & $ = \left(0, 0, \frac{p D}{p + q} - \frac{p k}{p + q} - a + p, 0\right)$\\
$\dots$ &\\
 &\\
($C_{p - 1}$) & $\left(\frac{a p + a q + n}{p} + k - p + 1, 0, \frac{p D}{p + q} - \frac{p k}{p + q} - a + p - 1, 0\right)$\\
 & $ = \left(0, \frac{q D}{p + q} + \frac{p k}{p + q} + a - p + 1, 0, p - 1\right)$\\
 &\\
($D_{p - 1}$) & $\left(q - \frac{a p + a q + n}{p} - k + 1, \frac{q D}{p + q} + \frac{p k}{p + q} + a - 1, 0, 0\right)$\\
 & $ = \left(0, 0, \frac{p D}{p + q} - \frac{p k}{p + q} - a + 1, p - 1\right)$\\
\end{tabular}
\end{center}
}

\end{center}
\end{frame}

%------------------------------------------------

\begin{frame}
\begin{center}
\frametitle{Frobenius Number}
{\footnotesize
Smaller $k$ Values:\\

$k < p - \frac{a p + a q + n}{p} + \frac{n}{d}$\\

~\\

$Max(Ap(S)) = \left(0, 0, \delta _ {p - 1}, p - 1\right)$\\

~\\

$F(S) = \delta _ {p - 1} a_3 + \left( p - 1 \right) a_4 - a_1$\\
$ = \left( p q - p - q \right) D + \left( q D - n \right) \left( \frac{p D - p k}{p + q} - a \right)$\\

~\\

~\\

Larger $k$ Values:\\

$k > p - \frac{a p + a q + n}{p} + \frac{n}{d}$\\

~\\

$Max(Ap(S)) = \left(0, \kappa _ {p - 1}, 0, p - 1\right)$\\

~\\

$F(S) = \kappa _ {p - 1} a_2 + \left( p - 1 \right) a_4 - a_1$\\
$ = \left( p q - p - q \right) D + \left( p D + n \right) \left( \frac{q D + p k}{p + q} + a - p \right)$}

\end{center}
\end{frame}

%------------------------------------------------
\section{Future Research}
%------------------------------------------------

\begin{frame}
\begin{center}
\frametitle{Future Research}

Future research:\\~\\
Expand proof into smaller values of $D$\\~\\
Develop new formula for very small values of $D$\\~\\
Balanced numerical semigroups with $6, 8, 10$ or more generators\\

\end{center}
\end{frame}

%------------------------------------------------
\section{Questions}
%------------------------------------------------

\begin{frame}
\begin{center}
\frametitle{Questions}

\Huge{\centerline{Questions?}}

\end{center}
\end{frame}

%------------------------------------------------
\section{}
%------------------------------------------------

\begin{frame}
\begin{center}
\frametitle{Thanks}

Research Partners:\\

~\\

Dr Kurt Herzinger, USAFA\\

~\\

Dr Trae Holcomb, Houston Baptist University

\end{center}
\end{frame}

\begin{frame}[noframenumbering]
\titlepage % Print the title page
\end{frame}

%------------------------------------------------

\begin{frame}[noframenumbering]
\begin{center}
\frametitle{$k$ - Congruence Class of $D$ Modulo $p + q$}

$k$ - the congruence class of $D$ modulo $p + q$\\

~\\

$k \in \{1 - \frac{a p + a q + n}{p}, 2 - \frac{a p + a q + n}{p}, \dots, p + q - \frac{a p + a q + n}{p}\}$\\
where $a = - n q ^ {-1} \bmod p$\\

~\\

\uncover<2- | handout : 0>{Example: $p = 2, q = 9, D = 8,$ and $n = 1$\\

~\\

$k \in \{ -5, -4, -3, -2, -1, 0, 1, 2, 3, 4, 5\}$\\
$k = -3$}

\end{center}
\end{frame}

%------------------------------------------------

\begin{frame}[noframenumbering]
\begin{center}
\frametitle{Frobenius Number of Balanced Numerical Semigroups}

We use relationships among the generators to find elements of $Ap(S)$\\

~\\

\uncover<2- | handout : 0>{If $x \in S$ and $x = t_1 a_1 + t_2 a_2 + t_3 a_3 + t_4 a_4$\\

~\\

We say $x = \left(t_1, t_2, t_3, t_4\right)$\\

~\\

$t_1, t_2, t_3, t_4 \ge 0$}

\end{center}
\end{frame}

%------------------------------------------------

\begin{frame}[noframenumbering]
\begin{center}
\frametitle{Relations Among the Generators}

Note that

{\footnotesize
\begin{center}
\begin{tabular}{cl}
($A$) & $\left(q, 0, 0, 0\right)=\left(0, 0, 0, p\right)$\\
\end{tabular}
\end{center}
}

~\\

implies\\

~\\

if $x = \left(0, 0, 0, a\right) \in Ap(S)$ then $a < p$

\end{center}
\end{frame}

%------------------------------------------------

\begin{frame}[noframenumbering]
\begin{center}
\frametitle{Relations Among the Generators}

{\scriptsize
\begin{center}
\begin{tabular}{cl}
($A$) & $\left(q, 0, 0, 0\right)=\left(0, 0, 0, p\right)$\\
 &\\
($B$) & $\left(1, 0, 0, 1\right) = \left(0, 1, 1, 0\right)$\\
 &\\
($C_0$) & $\left(\frac{a p + a q + n}{p} + k, 0, \frac{p D}{p + q} - \frac{p k}{p + q} - a, 0\right)$\\
 & $ = \left(0, \frac{q D}{p + q} + \frac{p k}{p + q} + a, 0, 0\right)$\\
 &\\
($D_0$) & $\left(q - \frac{a p + a q + n}{p} - k + p, \frac{q D}{p + q} + \frac{p k}{p + q} + a - p, 0, 0\right)$\\
 & $ = \left(0, 0, \frac{p D}{p + q} - \frac{p k}{p + q} - a + p, 0\right)$\\
$\dots$ &\\
 &\\
($C_{p - 1}$) & $\left(\frac{a p + a q + n}{p} + k - p + 1, 0, \frac{p D}{p + q} - \frac{p k}{p + q} - a + p - 1, 0\right)$\\
 & $ = \left(0, \frac{q D}{p + q} + \frac{p k}{p + q} + a - p + 1, 0, p - 1\right)$\\
 &\\
($D_{p - 1}$) & $\left(q - \frac{a p + a q + n}{p} - k + 1, \frac{q D}{p + q} + \frac{p k}{p + q} + a - 1, 0, 0\right)$\\
 & $ = \left(0, 0, \frac{p D}{p + q} - \frac{p k}{p + q} - a + 1, p - 1\right)$\\
\end{tabular}
\end{center}
}

\end{center}
\end{frame}

%------------------------------------------------

\begin{frame}[noframenumbering]
\begin{center}
\frametitle{Relations Among the Generators}

Note: These values only valid when\\

{\footnotesize
$k \in \{p - \frac{a p + a q + n}{p}, \dots, q - \frac{a p + a q + n}{p}\}$\\
$D \ge k + \frac{a p + a q}{p}, - \frac{k p}{q} - \frac{a p + a q}{q} + \frac{p}{q} \left( p + q \right)$\\}

~\\

\uncover<2- | handout : 0>{Similar arguments needed for\\
{\footnotesize
$k \in \{1 - \frac{a p + a q + n}{p}, \dots, p - 1 - \frac{a p + a q + n}{p}\}$\\}
and {\footnotesize
$k \in \{q + 1 - \frac{a p + a q + n}{p}, \dots, p + q - \frac{a p + a q + n}{p}\}$\\}}

~\\

\uncover<3- | handout : 0>{But ... results are the same for all values of $k$}

\end{center}
\end{frame}

%------------------------------------------------

\begin{frame}[noframenumbering]
\begin{center}
\frametitle{Apery Set Relative to $p D$}

Relations give us elements of $Ap(S)$\\

{\footnotesize
\begin{center}
\begin{tabular}{cl}
($A$) & $\left(0, 0, 0, a\right)$, $a \in \{ 0, \cdots, p - 1 \}$\\
 &\\
($C_0$) & $\left(0, c_0, 0, 0\right)$, $c_0 \in \{ 1, \cdots, \frac{q D}{p + q} + \frac{p k}{p + q} + a - 1 \}$\\
 &\\
($D_0$) & $\left(0, 0, d_0, 0\right)$, $d_0 \in \{ 1, \cdots, \frac{p D}{p + q} - \frac{p k}{p + q} - a + p - 1 \}$\\
 &\\
$\dots$ &\\
 &\\
($C_{p - 1}$) & $\left(0, c_{p - 1}, 0, p - 1\right)$, $c_{p - 1} \in \{ 1, \cdots, \frac{q D}{p + q} + \frac{p k}{p + q} + a - p \} $\\
 &\\
($D_{p - 1}$) & $\left(0, 0, d_{p - 1}, p - 1\right)$, $d_{p - 1} \in \{ 1, \cdots, \frac{p D}{p + q} - \frac{p k}{p + q} - a \} $\\
\end{tabular}
\end{center}
}

\end{center}
\end{frame}

%------------------------------------------------

\begin{frame}[noframenumbering]
\begin{center}
\frametitle{$Max(Ap(S))$}

$Max(Ap(S))$ is one of \\

{\footnotesize
\begin{center}
\begin{tabular}{cl}
($A$) & $\left(0, 0, 0, \alpha\right)$, $\alpha = p - 1$\\
 &\\
($C_0$) & $\left(0, \kappa _ 0, 0, 0\right)$, $\kappa _ 0 = \frac{q D}{p + q} + \frac{p k}{p + q} + a - 1$\\
 &\\
($D_0$) & $\left(0, 0, \delta _ 0, 0\right)$, $\delta _ 0 = \frac{p D}{p + q} - \frac{p k}{p + q} - a + p - 1$\\
 &\\
$\dots$ &\\
 &\\
($C_{p - 1}$) & $\left(0, \kappa _ {p - 1}, 0, p - 1\right)$, $\kappa _ {p - 1} = \frac{q D}{p + q} + \frac{p k}{p + q} + a - p$\\
 &\\
($D_{p - 1}$) & $\left(0, 0, \delta _ {p - 1}, p - 1\right)$, $\delta _ {p - 1} = \frac{p D}{p + q} - \frac{p k}{p + q} - a$\\
\end{tabular}
\end{center}
}

\end{center}
\end{frame}

%------------------------------------------------

\begin{frame}[noframenumbering]
\begin{center}
\frametitle{Frobenius Number}

Smaller $k$ Values:\\
{\footnotesize
$k < p - \frac{a p + a q + n}{p} + \frac{n}{d}$\\

~\\

$Max(Ap(S)) = \left(0, 0, \delta _ {p - 1}, p - 1\right)$\\

~\\

$F(S) = \delta _ {p - 1} a_3 + \left( p - 1 \right) a_4 - a_1$\\
$ = \left( p q - p - q \right) D + \left( q D - n \right) \left( \frac{p D - p k}{p + q} - a \right)$
}\\

~\\

~\\

Larger $k$ Values:\\
{\footnotesize
$k > p - \frac{a p + a q + n}{p} + \frac{n}{d}$\\

~\\

$Max(Ap(S)) = \left(0, \kappa _ {p - 1}, 0, p - 1\right)$\\

~\\

$F(S) = \kappa _ {p - 1} a_2 + \left( p - 1 \right) a_4 - a_1$\\
$ = \left( p q - p - q \right) D + \left( p D + n \right) \left( \frac{q D + p k}{p + q} + a - p \right)$
}

\end{center}
\end{frame}

%------------------------------------------------

\begin{frame}[noframenumbering]
\begin{center}
\frametitle{$D$ Ranges}

{\footnotesize
Middle $k$ Values:\\
$k \in \{p - \frac{a p + a q + n}{p}, \dots, q - \frac{a p + a q + n}{p}\}$\\
$D \ge k + \frac{a p + a q}{p}, - \frac{k p}{q} - \frac{a p + a q}{q} + \frac{p}{q} \left( p + q \right)$\\

~\\

Smaller $k$ Values:\\
$k \in \{1 - \frac{a p + a q + n}{p}, \dots, p - 1 - \frac{a p + a q + n}{p}\}$\\
$D \ge k + \frac{a p + a q}{p}, - \frac{k p}{q} - \frac{a p + a q}{q} + \frac{i}{q} \left( p + q \right)$\\
$i = k + \frac{a p + a q + n}{p}$\\

~\\

Larger $k$ Values:\\
$k \in \{q + 1 - \frac{a p + a q + n}{p}, \dots, p + q - \frac{a p + a q + n}{p}\}$
$D \ge k + \frac{a p + a q}{p} - \frac{p - i}{q} \left( p + q \right), - \frac{k p}{q} - \frac{a p + a q}{q} + \frac{p}{q} \left( p + q \right)$\\
$i = q + p - k - \frac{a p + a q + n}{p}$\\
}

\end{center}
\end{frame}

%------------------------------------------------

\begin{frame}[noframenumbering]
\begin{center}
\frametitle{Problem - Smaller $k$}

The relations $C_i, \cdots, C_{p - 1}$ have invalid $a_1$ values\\

~\\

$t_{C_i} = 0$ for $i = \frac{a p + a q + n}{p} + k$\

~\\

Can repair with relation ($B$)\\

~\\

$\left(1, 0, - 1, 1\right) = \left(0, 1, 0, 0\right)$

\end{center}
\end{frame}

%------------------------------------------------

\begin{frame}[noframenumbering]
\begin{center}
\frametitle{Relations Among the Generators - Smaller $k$}

{\tiny
\begin{center}
\begin{tabular}{cl}
($A$) & $\left(q, 0, 0, 0\right)=\left(0, 0, 0, p\right)$\\
 &\\
($C_0$) & $\left(\frac{a p + a q + n}{p} + k, 0, \frac{p D}{p + q} - \frac{p k}{p + q} - a , 0\right)$\\
 & $ = \left(0, \frac{q D}{p + q} + \frac{p k}{p + q} + a, 0, 0\right)$\\
 &\\
($D_0$) & $\left(q - \frac{a p + a q + n}{p} - k + p, \frac{q D}{p + q} + \frac{p k}{p + q} + a - p, 0, 0\right)$\\
 & $ = \left(0, 0, \frac{p D}{p + q} - \frac{p k}{p + q} - a + p, 0\right)$\\
$\cdots$ &\\
 &\\
($C_i ^ *$) & $\left(\frac{a p + a q + n}{p} + k - i + 1, 0, \frac{p D}{p + q} - \frac{p k}{p + q} - a + i - 1, 1\right)$\\
 & $ = \left(0, \frac{q D}{p + q} + \frac{p k}{p + q} + a - i + 1, 0, i\right)$\\
$\cdots$ &\\
 &\\
($C_{p - 1} ^ *$) & $\left(\frac{a p + a q + n}{p} + k - i + 1, 0, \frac{p D}{p + q} - \frac{p k}{p + q} - a + i - 1, p - i\right)$\\
 & $ = \left(0, \frac{q D}{p + q} + \frac{p k}{p + q} + a - i + 1, 0, p - 1\right)$\\
 &\\
($D_{p - 1}$) & $\left(q - \frac{a p + a q + n}{p} - k + 1, \frac{q D}{p + q} + \frac{p k}{p + q} + a - 1, 0, 0\right)$\\
 & $ = \left(0, 0, \frac{p D}{p + q} - \frac{p k}{p + q} - a + 1, p - 1\right)$\\
\end{tabular}
\end{center}
}

\end{center}
\end{frame}

%------------------------------------------------

\begin{frame}[noframenumbering]
\begin{center}
\frametitle{Apery Set - Smaller $k$}

The Apery Set can be represented as before\\

~\\

But we have duplicates and invalid elements as below:

{\footnotesize
\begin{center}
\begin{tabular}{cl}
($C_{i}$) & $\left(0, 0, \frac{p D}{p + q} - \frac{k}{p + q} - a + i , j\right)$\\
& $ = \left(0, \frac{q D}{p + q} + \frac{k}{p + q} + a - i, 0, i + j\right)$, $j \in \{ 0, \cdots, p - i - 1 \}$\\
 &\\
($C_{i + 1}$) & $\left(0, 0, \frac{p D}{p + q} - \frac{k}{p + q} - a + i + 1 , j\right)$\\
& $ = \left(1, \frac{q D}{p + q} + \frac{k}{p + q} + a - i + 1, 0, i + 1 + j\right)$, $j \in \{ 0, \cdots, p - i - 2 \}$\\
$\cdots$ &\\
 &\\
($C_{p - 1}$) & $\left(0, 0, \frac{p D}{p + q} - \frac{k}{p + q} - a + p - 1, j\right)$\\
& $ = \left(p - i, \frac{q D}{p + q} + \frac{k}{p + q} + a - p + 1, 0, p - 1 + j\right)$, $j \in \{ 0 \}$\\
\end{tabular}
\end{center}
}

\end{center}
\end{frame}

%------------------------------------------------

\begin{frame}[noframenumbering]
\begin{center}
\frametitle{Problem - Larger $k$}

The relations $D_i, \cdots, D_{p - 1}$ have invalid $a_1$ values\\

~\\

$t_{D_i} = 0$ for $i = q - \frac{a p + a q + n}{p} - k + p$\\

~\\

Can repair with relation ($B$)\\

~\\

$\left(1, - 1, 0, 1\right) = \left(0, 0, 1, 0\right)$

\end{center}
\end{frame}

%------------------------------------------------

\begin{frame}[noframenumbering]
\begin{center}
\frametitle{Relations Among the Generators - Larger $k$}

{\tiny
\begin{center}
\begin{tabular}{cl}
($A$) & $\left(q, 0, 0, 0\right)=\left(0, 0, 0, p\right)$\\
 &\\
($C_0$) & $\left(\frac{a p + a q + n}{p} + k, 0, \frac{p D}{p + q} - \frac{p k}{p + q} - a, 0\right)$\\
 & $ = \left(0, \frac{q D}{p + q} + \frac{p k}{p + q} + a, 0, 0\right)$\\
 &\\
($D_0$) & $\left(q - \frac{a p + a q + n}{p} - k + p, \frac{q D}{p + q} + \frac{p k}{p + q} + a - p, 0, 0\right)$\\
 & $ = \left(0, 0, \frac{p D}{p + q} - \frac{p k}{p + q} - a + p, 0\right)$\\
$\cdots$ &\\
 &\\
($D_i ^ *$) & $\left(q - \frac{a p + a q + n}{p} - k + p - i + 1, \frac{q D}{p + q} + \frac{p k}{p + q} + a - p + i - 1, 0, 1\right)$\\
 & $ = \left(0, 0, \frac{p D}{p + q} - \frac{p k}{p + q} - a + p - i + 1, i\right)$\\
$\cdots$ &\\
 &\\
($C_{p - 1}$) & $\left(\frac{a p + a q + n}{p} + k - p + 1, 0, \frac{p D}{p + q} - \frac{p k}{p + q} - a + p - 1, 0\right)$\\
 & $ = \left(0, \frac{q D}{p + q} + \frac{p k}{p + q} + a - p + 1, 0, p - 1\right)$\\
 &\\
($D_{p - 1} ^ *$) & $\left(q - \frac{a p + a q + n}{p} - k + p - i + 1, \frac{q D}{p + q} + \frac{p k}{p + q} + a - p + i - 1, 0, p - i\right)$\\
 & $ = \left(0, 0, \frac{p D}{p + q} - \frac{p k}{p + q} - a + p - i + 1, p - 1\right)$\\
\end{tabular}
\end{center}
}

\end{center}
\end{frame}

%------------------------------------------------

\begin{frame}[noframenumbering]
\begin{center}
\frametitle{Apery Set - Larger $k$}

The Apery Set can be represented as before\\

~\\

But we have duplicates and invalid elements as below:

{\footnotesize
\begin{center}
\begin{tabular}{cl}
($D_i$) & $\left(0, \frac{q D}{p + q} + \frac{k}{p + q} + a - p + i, 0, j\right)$\\
& $ = \left(0, 0, \frac{p D}{p + q} - \frac{k}{p + q} - a + p - i, i + j\right)$, $j \in \{ 0, \cdots, p - i - 1\}$\\
 &\\
($D_{i + 1}$) & $\left(0, \frac{q D}{p + q} + \frac{k}{p + q} + a - p + i + 1, 0, j\right)$\\
& $ = \left(1, 0, \frac{p D}{p + q} - \frac{k}{p + q} - a + p - i - 1, i + 1 + j\right)$, $j \in \{ 0, \cdots, p - i - 2 \}$\\
$\cdots$ &\\
 &\\
($D_{p - 1}$) & $\left(0, \frac{q D}{p + q} + \frac{k}{p + q} + a - 1, 0, j\right)$\\
& $ = \left(p - i, 0, \frac{p D}{p + q} - \frac{k}{p + q} - a + 1, p - 1 + j\right)$, $j \in \{ 0 \}$\\
\end{tabular}
\end{center}
}

\end{center}
\end{frame}

%------------------------------------------------

\begin{frame}[noframenumbering]
\begin{center}
\frametitle{Frobenius Number}

Smaller $k$ Values:\\
{\footnotesize
$k < p - \frac{a p + a q + n}{p} + \frac{n}{d}$\\

~\\

$Max(Ap(S)) = \left(0, 0, \delta _ {p - 1}, p - 1\right)$\\

~\\

$F(S) = \delta _ {p - 1} a_3 + \left( p - 1 \right) a_4 - a_1$\\
$ = \left( p q - p - q \right) D + \left( q D - n \right) \left( \frac{p D - p k}{p + q} - a \right)$
}\\

~\\

~\\

Larger $k$ Values:\\
{\footnotesize
$k > p - \frac{a p + a q + n}{p} + \frac{n}{d}$\\

~\\

$Max(Ap(S)) = \left(0, \kappa _ {p - 1}, 0, p - 1\right)$\\

~\\

$F(S) = \kappa _ {p - 1} a_2 + \left( p - 1 \right) a_4 - a_1$\\
$ = \left( p q - p - q \right) D + \left( p D + n \right) \left( \frac{q D + p k}{p + q} + a - p \right)$
}

\end{center}
\end{frame}

%------------------------------------------------

\begin{frame}[noframenumbering]
\titlepage % Print the title page
\end{frame}

%------------------------------------------------

%----------------------------------------------------------------------------------------

\end{document} 